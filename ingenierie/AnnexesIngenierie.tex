%Annexes du rapport d'ingénierie



\chapter{Librairies utilisées pour le projet de Comparaison d'images}


\paragraph{La librairie "Insight ToolKit" (ITK)} est un projet open source initialement destiné au recalage et a la segmentation d'images medicales. ITK a été programmée en \C++ , en utilisant des techniques de codage avancées (templates, modification de la syntaxe standard et ajout de fonctionnalités au langage \C++ par le biais d'idiomes) , ainsi que l'outil de développement cross-platform CMake, et communautaires (SVN, puis GIT). 
Cette librairies est construite sur un système de "Templates", qui lui permet de s'adapter à diverses données. Elle propose une architecture centrées sur un flux de données, traité par différents filtres que l'on connecte ensemble.
Cette collection d'algorithmes ne cesse de s'agrandir grâce à la philosophie open-source. Le spectre des applications d'ITK inclut entre autres exemples:  
\begin{itemize}
  \item l'imagerie medicale, avec \href{http://www.slicer.org/}{3DSlicer},
  \item l'imagerie biologique, avec \href{http://gofigure2.sourceforge.net/}{Gofigure2},
  \item et l'imagerie satellite, avec l'\href{http://www.orfeo-toolbox.org/otb/}{Orfeo ToolBox}.
\end{itemize}
Le modèle de programmation est relativement complexe et nécessite de un long apprentissage, et de l'expérience. A cette fin, de nombreux outils d'information sont mis a la disponibilité du nouvel utilisateur :
\begin{itemize}
  \item des listes de diffusions d'Emails, pour mettre en contact les nouveaux utilisateurs d'ITK et les programmeurs et utilisateurs avancés;
  \item un wiki (\url{http://www.vtk.org/Wiki/ITK}), qui donne quelques informations quand à la mise en place d'un environnement de développement utilisant ITK.;
  \item un guide d'utilisateur très bien expliqué, mais basé sur la version 2.4 d'ITK tandis que la dernière version publiée est la 3.20;
    \TODO {reference ITK software guide}
  \item la documentation Doxygen présentée sous forme de site internet, est directement compilée a partir du code source. Elle est donc plus ou moins complète selon les fichiers. Afin de pouvoir naviguer dans cette dernière, il est indispensable de connaitre la structure du projet.
\end{itemize}
Luis Ibáñez, a dit cette année : "La courbe d'apprentissage d'ITK en \C++ est bien trop raide, et nous allons nous efforcer dans les futures version, de rendre la librairie plus accessible.". La prochaine version d'ITK (version 4) sera donc surement plus facile a utiliser et prendre en main.

\paragraph{La librairie VTK} est développée par Kitware, Il s'agit d'une librairie \C++ utilisee pour la visualisation de données. Elle est open source et cross-platform. Elle utilise des outils similaires a ceux utilises pour ITK (CMake)
Elle utilise un systeme de "pipeline" (flux de donnees) similaire a celui d'ITK pour traiter les donnees a visualiser. Elle est developpee conjointement avec ITK, et il est possible plus ou moins facilement de connecter les filtres de traitement d'image d'ITK avec les filtres de visualisation de VTK.

\paragraph{La librairie Qt} permet une gestion avancée de l'interface utilisateur, en proposant une interface graphique open source et cross-platform.
Elle étend aussi les fonctionnalités du langage \C++ en proposant une nouvelle architecture pour le système de callbacks, un nouveau modèle d'objet, et d'héritage. Cependant, ces modifications sont facilement intégrées par le développeur débutant, grâce a une aide abondante composée d'un ensemble de tutoriels, d'exemples fortement documentes, et d'une liste de diffusion très active.







\chapter{Définitions}


Templates : En programmation informatique, les templates sont une particularité de la programmation en language C++, qui autorise l'écriture d'un code sans considération envers le type des données avec lesquelles il sera finalement utilisé. Les templates supportent la programmation générique en {\C++}.

Cross-platform : Un logiciel multiplate-forme ou multiplateforme est un logiciel conçu pour fonctionner sur plusieurs plates-formes, c’est-à-dire le couple liant ordinateur et système d’exploitation. En anglais on parle souvent de "cross-platform software" ou "platform independent software" ou encore de "multi-platform software".

Idiomes (programmation) :  Un idiome en programmation qualifie un code qui ajoute des fonctionnalités non existantes dans un
langage.

Wiki : Un wiki est un site web dont les pages sont modifiables par tout ou partie des visiteurs du site. Il permet ainsi l'écriture et l'illustration collaboratives de documents.

Graphical User Interface (GUI) : Un environnement graphique est, en informatique, ce qui est affiché en pixels sur un moniteur
d'ordinateur et sur lequel l'utilisateur peut agir avec différents périphériques d'entrée comme le clavier ou la souris. 
Des images, des animations (en 2 ou 3 dimensions), et même des vidéos peuvent être rendues à l'écran.
Ce type d'interface Homme-machine s'oppose à la notion de ligne de commande.

Callback : la technique des fonctions de rappel (callback functions) permet de passer en argument d'une fonction, une autre fonction. 
Cette technique est utilisée dans la programmation évènementielle, ou les interactions de l'utilisateur doivent entrainer l'exécution de fonctions.

\chapter{Biblio TODO}

Evaluation/comparaisons de GIT :
descritpion comparaison centralisee non centralisee
\url{http://informatique.in2p3.fr/?q=node/333}
avantages de GIT :
\url{http://www.whygitisbetterthanx.com/}
\url{http://joshcarter.com/productivity/svn_hg_git_for_home_directory}
\url{http://dev-heaven.net/wiki/20/Git_vs_SVN_comparison}

Matt Mc Cornick

