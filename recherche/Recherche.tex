% Chapitre sur le rapport de recherche :


\chapter{Rapport de recherche} 

\section*{Introduction}

\subsection*{}
Je présente deux dans cette partie, la démarche recherche que j'ai eue pendant le PFE. Il s'agit tout d'abord de bien prendre connaissance du sujet. Cette prise de connaissance permet de voir quels problèmes persistent. Il faut ensuite trouver une solution à ces problèmes, afin de pouvoir avancer dans sa résolution.

La difficulté en recherche, est que bien souvent, il n'existe pas de solution dans le domaine étudié.
Il faut donc passer par une phase de bibliographie, pour étudier l'état de l'art,
et éventuellement apprendre de nouvelles théories.
Le domaine du traitement de l'image est une science particulière car beaucoup d'algorithmes sont inventés,
mais très peu sont diffusés ou applicable a divers types d'images.
Il faut donc reprogrammer les algorithmes présentés dans les publications scientifiques et vérifier leur performance.
Il faut enfin trouver des idées innovantes pour résoudre les problèmes posés, et expérimenter.

\subsection*{}
J'ai travaillé, pendant mon PFE au Megason Lab sur la segmentation d'images obtenues par microscopie fluorescente.
Les données sont quadri-dimensionnelles (espace et temps), et représentent des régions du poisson zèbre (oreille, colonne vertébrale...). Il s'agit de vidéos prises pendant le développement du spécimen. Nous pouvons bien discerner les différentes cellules du poisson, ainsi que leur noyaux. Ces éléments sont donc la base du modèle que nous tentons de créer. Il faut donc être en mesure de détecter et suivre ces cellules au cours du temps. Le modèle devra aussi intégrer les informations morphologiques de chaque cellule.

Ce modèle fait l'objet d'une proposition de thèse : Modèle numérique cellulaire de poisson zèbre ( cells lineage registration in microscopy) durant laquelle j'aimerais poursuivre mes travaux.

Avant d'arriver au Laboratoire, j'ai travaillé sur la membrane cellulaire. J'ai ensuite concentré mes recherches sur la détection et la localisation de noyaux.

\section{Segmentation de la paroi cellulaire par ensembles de niveaux}

Le but initial du PFE etait la segmentation de la paroi cellulaire. il s'agit d'une fine membrane séparant les multiples cellules. Elle s'étend sur tout le spécimen a analyser. Il s'agit donc d'un volume important et complexe.

\subsection{Étude du problème}

J'ai tout d'abord cherché à comprendre le problème posé : sur quelles données allaient se baser la détection, existe-t'il des solutions pour segmenter ce genre de données.

\subsubsection{Les données}
Les images sont acquises a travers un système optique. L'excitation par un laser entraine la fluorescence de certaines parties de la cellule, marquées par une molécule émettant de la lumière dans un spectre dépendant du marqueur utilisé.

Le système a donc une réponse impulsionelle bien visible dans les données. Un point correspond grossièrement a une gaussienne étalée dans les trois dimensions de l'espace, et plus particulièrement selon l'axe perpendiculaire au plan de focalisation.

Il existe aussi un bruit du au dispositif électronique d'acquisition. De plus, la fluorescence n'étant pas repartie de manière homogène, il existe des "trous" et de la saturation dans les données.
\TODO{inserer des images illustrant les problemes}


J'ai choisi de me focaliser sur trois difficultés afin de trouver des solutions :
\begin{description}
  \item [problème du bruit] : quel filtrage appliquer aux images, afin de les débruiter.
  \item [problème de l'absence de données] : comment introduire des à priori de forme de la membrane
  pour palier à l'absence d'information ?
  \item [problème de la non homogénéité des intensités]  : comment segmenter un objet
  qui n'occupe pas les mêmes intensités selon sa position dans l'espace.
\end{description}


\subsection{Utilisation de la théorie des ensembles de niveaux}

L'outil choisi pour segmenter la paroi cellulaire, est base sur les ensembles de niveaux (levels-sets). Cette théorie consiste en l'évolution d'un front. Cette évolution est représentée par une fonction implicite qui évolue itérativement. Le front (bords de la zone segmentée) est souvent représenté par le niveau zéro de cette fonction implicite.
Les Level sets, au travers de leur critère d'évolution, permettent d'avoir une grande flexibilité quand aux mesures a considerer lors de l'evolution du front. Cette évolution est représentée par un critère d'énergie, le problème de segmentation par level set est donc un problème d'optimisation.

\subsection{des idees}



Afin d'éliminer le bruit sans détériorer la membrane, le principe de l'érosion dilatation a été mis en oeuvre. Il s'agit d'une methode rapide qui donne de bons resultats


idee de la mediane
idee morphologie
idee localisation
\subsection{resultats}
idee mediane
idee morphologie
idee localisation
\subsection{travail futur}
rapidite



\section{Detection et localisation des cellules}

Nous basons nos méthodes de segmentation sur une initialisation au centre des cellules. Nous avons donc besoin de détecter un maximum de cellules afin de trouver un point a l'intérieur de ces dernières. Des methodes ont ete proposees, cependant, chacune est adaptee a un type d'image particulier.
Cels algorithmes de détection sont aussi souvent appelles algorithmes de "seeding" car ils permettent d'obtenir des points a partir desquels une segmentation peut etre initialisee, afin de delimiter les bordures des noyaux, ou les membranes cellulaires.

\subsection{demarche}

Nous avons developpe une methode combinant l'information provenant des noyaux et de la membrane des cellules. Cette methode doit etre evaluee, donc comparee a d'autres methodes existantes. Ce processus d'evaluation nous permettra aussi de trouver les points forts et les points faibles des algorithmes. Nous pourrons ainsi eventuellement utiliser des techniques de fusion d'information pour combiner les resultats de differents algorithmes.
La creation d'un "framework" d'evaluation passe donc par plusieurs etapes : l'implementation des algorithmes existants, afin de les tester sur des images synthetiques puis reelles, la creation de criteres d'evaluation appropories, et l'observation des resultats. Nous avons aussi initié un travail afin de proposer une nouvelle methode de detection de cellules basee sur la decomposition en ondelettes.


\subsection {description des algorithmes evalues}



\subsubsection{chaine de traitement de l'image}

partie commune
Nous nous focalisons sur une classe d'algorithmes traitant l'information issue de l'image des noyaux cellulaires, apres une detection des zones d'interet (binarisation de l'image). Ces algorithmes fonctionnent aussi souvent avec une extraction de maxima locaux en dernier traitement.
Nous choisissons d'utiliser la même binarisation, et la meme methode d'extraction de maximas locaux pour les deux algorithmes afin de focaliser l'etude sur la technique de detection des centres des noyaux.

\subsubsection{description des algorithmes}
\paragraph{le Laplacien de la Gaussienne ameliore}
Nous avons decide d'implementer l'algorithme presente dans \cite{al2009improved}. La methode utilisee est celle du Laplacien de la Gaussienne (LoG). Une methode eprouveee qui s'est montree tres robuste dans d'autres applications telles la detection de points de reperes pour le recallage photographique.



\paragraph{Kishore}

\TODO{Ask Kishore more infos}


\subsection {evaluation}
%\begin{tabular}{|c|c|c|c|c|}
%\hline  & Matching & UnMatching & Missed & Accuracy \\ 
%\hline A1 & 10 & 3 & 1 & 71% \\ 
%\hline A2 & 9 & 2 & 3 & 64% \\ 
%\hline 
%\end{tabular} 



\subsection {conclusion}


\subsection {proposition}



\subsection {planning}


\subsection{resultats}

\subsection{proposition}




